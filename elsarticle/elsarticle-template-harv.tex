%% 
%% Copyright 2007-2020 Elsevier Ltd
%% 
%% This file is part of the 'Elsarticle Bundle'.
%% ---------------------------------------------
%% 
%% It may be distributed under the conditions of the LaTeX Project Public
%% License, either version 1.2 of this license or (at your option) any
%% later version.  The latest version of this license is in
%%    http://www.latex-project.org/lppl.txt
%% and version 1.2 or later is part of all distributions of LaTeX
%% version 1999/12/01 or later.
%% 
%% The list of all files belonging to the 'Elsarticle Bundle' is
%% given in the file `manifest.txt'.
%% 
%% Template article for Elsevier's document class `elsarticle'
%% with harvard style bibliographic references

\documentclass[preprint,12pt,authoryear]{elsarticle}

%% Use the option review to obtain double line spacing
%% \documentclass[authoryear,preprint,review,12pt]{elsarticle}

%% Use the options 1p,twocolumn; 3p; 3p,twocolumn; 5p; or 5p,twocolumn
%% for a journal layout:
%% \documentclass[final,1p,times,authoryear]{elsarticle}
%% \documentclass[final,1p,times,twocolumn,authoryear]{elsarticle}
%% \documentclass[final,3p,times,authoryear]{elsarticle}
%% \documentclass[final,3p,times,twocolumn,authoryear]{elsarticle}
%% \documentclass[final,5p,times,authoryear]{elsarticle}
%% \documentclass[final,5p,times,twocolumn,authoryear]{elsarticle}

%% For including figures, graphicx.sty has been loaded in
%% elsarticle.cls. If you prefer to use the old commands
%% please give \usepackage{epsfig}

%% The amssymb package provides various useful mathematical symbols
\usepackage{amsthm}

\usepackage{amsmath}

\usepackage{booktabs} % 导入三线表需要的宏包
\usepackage{longtable}% 导入跨页表格所需宏包

%% The lineno packages adds line numbers. Start line numbering with
%% \begin{linenumbers}, end it with \end{linenumbers}. Or switch it on
%% for the whole article with \linenumbers.
%% \usepackage{lineno}

\journal{Nuclear Physics B}

\begin{document}

\begin{frontmatter}

%% Title, authors and addresses

%% use the tnoteref command within \title for footnotes;
%% use the tnotetext command for theassociated footnote;
%% use the fnref command within \author or \affiliation for footnotes;
%% use the fntext command for theassociated footnote;
%% use the corref command within \author for corresponding author footnotes;
%% use the cortext command for theassociated footnote;
%% use the ead command for the email address,
%% and the form \ead[url] for the home page:
%% \title{Title\tnoteref{label1}}
%% \tnotetext[label1]{}
%% \author{Name\corref{cor1}\fnref{label2}}
%% \ead{email address}
%% \ead[url]{home page}
%% \fntext[label2]{}
%% \cortext[cor1]{}
%% \affiliation{organization={},
%%            addressline={}, 
%%            city={},
%%            postcode={}, 
%%            state={},
%%            country={}}
%% \fntext[label3]{}

\title{Parallel vehicle scheduling 
with compatibility constraints and makespan objectives: 
An application to automated guided vehicle  scheduling}

%% use optional labels to link authors explicitly to addresses:
%% \author[label1,label2]{}
%% \affiliation[label1]{organization={},
%%             addressline={},
%%             city={},
%%             postcode={},
%%             state={},
%%             country={}}
%%
%% \affiliation[label2]{organization={},
%%             addressline={},
%%             city={},
%%             postcode={},
%%             state={},
%%             country={}}

\author{Mo Di}

\affiliation{organization={},%Department and Organization
            addressline={}, 
            city={},
            postcode={}, 
            state={},
            country={}}

\begin{abstract}
%% Text of abstract

\end{abstract}

%%Graphical abstract
\begin{graphicalabstract}
%\includegraphics{grabs}
\end{graphicalabstract}

%%Research highlights
\begin{highlights}
\item Research highlight 1
\item Research highlight 2
\end{highlights}

\begin{keyword}
    Scheduling problem \sep
    Minimum makespan \sep
    Eligibility constraints \sep
    Automated guided vehicle \sep
    Benders decomposition
%% keywords here, in the form: keyword \sep keyword

%% PACS codes here, in the form: \PACS code \sep code

%% MSC codes here, in the form: \MSC code \sep code
%% or \MSC[2008] code \sep code (2000 is the default)

\end{keyword}

\end{frontmatter}

%% \linenumbers

%% main text
\section{Introduction}

In this paper, a scheduling problem of multiple vehicles and product types is investigated. 
The problem has the  characteristics of both vehicle problems with pickups and deliveries,
and unrelated and uniform parallel machine scheduling problems.


\section{Literature review}

\subsection{Vehicle routing problem to minimize makespan}

Vehicle Routing Problems with Pickups and Deliveries

The Vehicle Routing Problem (VRP) is a classical optimization problem, 
which aims to select the optimal route for vehicles based on one or more depots,
under specific conditions (i.e. time windows, capacity constraints).

Various exact or heuristic algorithms are developed to solve different types of VRPs.

\subsection{Unrelated and uniform parallel machine scheduling}

As another famous optimization problem, Machine Scheduling (MS) 
is widely applied in manufacturing, logistics.

$R_r||C_{\max}$                                                                                                                                                      

\subsection{Summary and contributions of the paper}

\section{Problem formulation}

In this section, we first formally define the parallel vehicle scheduling problem 
with compatibility constraints and makespan objectives, 
and then propose MILP models.

\subsection{Problem definition}

To be easier to describe this problem, the notations used in this paper are listed in Table \ref{tab1}.

\subsection{Consumption and preprocessing}

\subsection{Mathematical formulation}

\begin{longtable}{cl}% 其中,tabular是表格内容的环境;c表示centering,即文本格式居中;c的个数代表列的个数
    \caption{Notations}\label{tab1}\\
    \toprule [2pt] % 设置线宽
    sets& definitions \\
    \midrule [1pt]  
    $K$& set of vehicles \\
    $V$& set of machines \\
    $W$& set of product type \\
    $D$& set of   production plan,
    $D=\{(i,j)|i,j\in V\backslash \{v_0\}\}$\\
    $D_w$& set of production plan with particular product type, $w \in W$ \\
    $R$& set of gernal task sequence length, $R={1,2,...}$ \\
    \midrule [1pt]     
    parameters & definitions  \\ %换行
    \midrule [1pt]
    $v_0$& the depot, $v_0\in V$ \\
    $c_k$& capacity of vehicles $k$, $k\in K$ \\
    $s_i^k$& servcing time of vehicle $k$ for machine $i$, $i\in V,k\in K$ \\
    $t_{ij}$& travel cost from machine $i$ to $j$, $i,j\in V$ \\ 
    $q_{ij}$& amount of demand $(i,j)$, $(i,j)\in D$ \\
    $T_m$& maximum makespan \\
    \midrule[1pt]
    variables & definitions \\
    \midrule[1pt]
    $u_w^k$& whether vehicle $k$ loading products of $w$ type \\
    $x_{ir}^k$& whether vehicle $k$ serve machine $i$ as its $r$-th task \\
    $\psi_{ir}^k$& reserve of last task of vehicle $k$ in $r$-th one \\
    $\epsilon_r^k$& waiting time of vehicle $k$ before $r$-th task \\
    $g_{ir}$& avaliable time of machine $i$ in $r$-th task \\
    $T_f$& gernal finish time \\
    & $w\in W,i,j\in V,k\in K,r\in R$\\
    \bottomrule [2pt]
\end{longtable}

Unrelated parallel machine scheduling problem with the objective of 
minimizing the maximal completion time($R||C_{\max}$) and restrict PDP

objective
\begin{equation}
    \min T_f
\label{msobj}
\end{equation}

subject to

orgin and destination
\begin{equation}
    x_{v_0,0}^k,x_{v_0,|R|+1}^k=1,\forall k\in K 
\label{od}
\end{equation}

\begin{equation}
    x_{v_0,r}^k,x_{i,0}^k=0,\forall i\in V, k\in K,r\in R 
\label{dpt}
\end{equation}

single occupied
\begin{equation}
    \sum_{i\in V,k\in K}{x_{ir}^k}\leq 1, \forall r\in R
\label{so}
\end{equation}

\begin{equation}
    \sum_{i\in V,k\in K}{x_{ir}^k}\geq \sum_{i\in V,k\in K}{x_{i,r+1}^k}, \forall r\in R
\label{zeroafter}
\end{equation}

single vehicle sequence
\begin{equation}
    \psi_{ir}^k=x_{ir}^k+\psi_{i,r-1}^k
    \cdot (1-\sum_{j\in V}{x_{jr}^k}),\forall i\in V,k\in K,r\in R
\label{svs}
\end{equation}
\begin{equation}
    \begin{split}
        g_{i,r-1}-T_m\cdot \sum_{k\in K}{x_{ir}^k} \leq g_{ir}
        \leq g_{i,r-1}+T_m\cdot \sum_{k\in K}{x_{ir}^k},\\
        \forall i\in V ,r\in R
    \end{split}
\label{god}
\end{equation}
\begin{equation}
    \begin{split}
        -T_m\cdot \sum_{i\in V}{x_{ir}^k} \leq \epsilon_r^k \leq T_m\cdot \sum_{i\in V}{x_{ir}^k},
        \forall r\in R,k\in K
    \end{split}
\label{eod}
\end{equation}

compatibility constraints
\begin{equation}
    \sum_{w\in W}{u_w^k}\leq 1, \forall k\in K
\label{scc}
\end{equation}

\begin{equation}
    \psi_{i,r-1}^k \cdot x_{jr}^k\leq u_w^k, \forall (i,j)\in D_w, w\in W, r\in R, k\in K
\label{cc}
\end{equation}

time
\begin{equation}
    \sum_{r\in R}{(
        \sum_{i\in V}{s_i^k\cdot x_{i,r-1}^k}
        +\sum_{i,j\in V}{t_{ij}\cdot \psi_{i,r-1}^k\cdot x_{jr}^k}
        +\epsilon_r^k
    )}
    \leq T_f,
    \forall k\in K
\label{tm}
\end{equation}

fulfill demand
\begin{equation}
    \sum_{k\in K,r\in R}{c_k\cdot \psi_{i,r-1}^k\cdot x_{jr}^k}\geq q_{ij}, \forall (i,j)\in D
\label{fd}
\end{equation}

one by one
\begin{equation}
    \begin{split}
        \sum_{r\in \{1,..,\gamma\}}{(
            \sum_{i\in V}{s_i^k\cdot x_{i,r-1}^k}
            +\sum_{i,j\in V}{t_{ij}\cdot \psi_{i,r-1}^k\cdot x_{jr}^k}
            +\epsilon_r^k
        )} \geq \\
        \sum_{i\in V}{g_{i,\gamma}\cdot x_{i,\gamma}^k}-T_m\cdot(1-\sum_{i\in V}{x_{ir}^k}),
        \forall k\in K,\gamma \in R
    \end{split}
\label{1b1}
\end{equation}

avaliable time
\begin{equation}
    \begin{split}
        &\sum_{i\in V}{g_{i,\gamma+1}\cdot x_{i,\gamma}^k}-T_m\cdot(1-\sum_{i\in V}{x_{ir}^k})\\
        &\leq \sum_{r\in \{1,..,\gamma\}}{(
            \sum_{i\in V}{s_i^k\cdot x_{i,r-1}^k}
            +\sum_{i,j\in V}{t_{ij}\cdot \psi_{i,r-1}^k\cdot x_{jr}^k}
            +\epsilon_r^k
        )} + \sum_{i\in V}{s_i^k\cdot x_{i,r}^k}\\
        &\leq \sum_{i\in V}{g_{i,\gamma+1}\cdot x_{i,\gamma}^k} + T_m\cdot(1-\sum_{i\in V}{x_{ir}^k}),
        \forall k\in K,\gamma \in R
    \end{split}
\label{avlt}
\end{equation}

\begin{equation}
    g_{ir}\leq T_m, \forall i\in V,r\in R
\label{mxms}
\end{equation}

domain of definition
\begin{equation}
    u_k^w,x_{ir}^k\in \{0,1\}, \forall w\in W, i\in V,r\in R,k\in K
\label{ddb}
\end{equation}

\begin{equation}
    \epsilon_r^k,g_{ir},\tau_s^k,T_f\geq 0,\forall i\in V, k\in K,r\in R
\label{ddc}
\end{equation}

Add $Y,Z,\Pi$
\begin{longtable}{cl}% 其中,tabular是表格内容的环境;c表示centering,即文本格式居中;c的个数代表列的个数
    \caption{Instrumental variables}\label{tab2}\\
    \toprule [2pt] % 设置线宽
    variables& definitions \\
    \midrule [1pt]  
    $y_{ijr}^k$& $\psi_{i,r-1}^k\cdot x_{jr}^k$ \\
    $z_{ir}^k$& $g_{ir}\cdot x_{ir}^k$ \\
    $\pi_{ir}^k$& $g_{i,r+1}\cdot x_{ir}^k$ \\
    &$\forall i,j\in V, r\in R, k\in K$ \\
    \bottomrule [2pt]
\end{longtable}

convert nonlinear to linear

\begin{equation}
    \begin{split}
        \begin{cases}
            & y_{ijr}^k\leq \psi_{i,r-1}^k \\
            & y_{ijr}^k\leq x_{j,r}^k \\
            & y_{ijr}^k\geq \psi_{i,r-1}^k+x_{j,r}^k-1
        \end{cases}, \forall i,j\in V,k\in K,r\in R
    \end{split}
    \label{nly}
\end{equation}

\begin{equation}
    \begin{split}
        \begin{cases}
            & z_{ir}^k\leq T_m\cdot x_{ir}^k \\
            & z_{ir}^k\leq g_{ir}\\
            & z_{ir}^k\geq T_m\cdot(x_{ir}^k -1) + g_{ir}
        \end{cases}, \forall i\in V, r\in R, k\in K
    \end{split}
    \label{nlz}
\end{equation}

\begin{equation}
    \begin{split}
        \begin{cases}
            & \pi_{ir}^k\leq T_m\cdot x_{ir}^k \\
            & \pi_{ir}^k\leq g_{i,r+1} \\
            & \pi_{ir}^k\geq T_m\cdot(x_{ir}^k -1) + g_{i,r+1}
        \end{cases},  \forall i\in V, r\in R, k\in K
    \end{split}
    \label{nlpi}
\end{equation}

single vehicle sequence
\begin{equation}
    \psi_{ir}^k=x_{ir}^k+\psi_{i,r-1}^k-\sum_{j\in V}{y_{ijr}^k},\forall i\in V,k\in K,r\in R
\label{svsy}
\end{equation}

compatibility constraints
\begin{equation}
    y_{ijr}^k\leq u_w^k, \forall (i,j)\in D_w, r\in R, k\in K
\label{ccy}
\end{equation}

time
\begin{equation}
    \sum_{r\in R}{(
        \sum_{i\in V}{s_i^k\cdot x_{i,r-1}^k}
        +\sum_{i,j\in V}{t_{ij}\cdot y_{ijr}^k}
        +\epsilon_r^k
    )}
    \leq T_f,
    \forall k\in K
\label{tmy}
\end{equation}

fulfill demand
\begin{equation}
    \sum_{k\in K,r\in R}{c_k\cdot y_{ijr}^k}\geq q_{ij}, \forall (i,j)\in D
\label{fdy}
\end{equation}

one by one
\begin{equation}
    \begin{split}
        \sum_{r\in \{1,..,\gamma\}}{(
            \sum_{i\in V}{s_i^k\cdot x_{i,r-1}^k}
            +\sum_{i,j\in V}{t_{ij}\cdot y_{ijr}^k}
            +\epsilon_r^k
        )} \geq \\
        \sum_{i\in V}{z_{ir}^k}+T_m\cdot(1-\sum_{i\in V}{x_{ir}^k}),
        \forall k\in K,\gamma \in R
    \end{split}
\label{1b1y}
\end{equation}

avaliable time
\begin{equation}
    \begin{split}
        &\sum_{i\in V}{\pi_{ir}^k}-T_m\cdot(1-\sum_{i\in V}{x_{ir}^k})\\
        &\leq \sum_{r\in \{1,..,\gamma\}}{(
            \sum_{i\in V}{s_i^k\cdot x_{i,r-1}^k}
            +\sum_{i,j\in V}{t_{ij}\cdot y_{ijr}^k}
            +\epsilon_r^k
        )}  + \sum_{i\in V}{s_i^k\cdot x_{i,r}^k}\\
        &\leq \sum_{i\in V}{\pi_{ir}^k}+T_m\cdot(1-\sum_{i\in V}{x_{ir}^k}),
        \forall k\in K,\gamma \in R
    \end{split}
\label{avltyz}
\end{equation}

domain of definition
\begin{equation}
    y_{ijr}^k\in \{0,1\}, \forall i,j\in V,r\in R,k\in K
\label{ddy}
\end{equation}

\begin{equation}
    z_{ir}^k,\pi_{ir}^k\geq 0, \forall i\in V,r\in R,k\in K
\label{ddz}
\end{equation}


\section{XXXX Algorithm}

Heuristic Benders algorithm

master-problem (MP):
$U,X,\Psi,Y$
\begin{equation}
    \min 0+\overline{T_f}
\label{mobj}
\end{equation}

s.t. (\ref{od}), (\ref{dpt}), (\ref{so}), (\ref{zeroafter}), (\ref{nly}), 
(\ref{svsy}), (\ref{scc}), (\ref{ccy}), (\ref{fdy}) and cuts from sub-problem.

There is no objection in master-problem, just find the feasible solution.

sub-problem (SP):
$E,G,T_f,Z,\Pi$
\begin{equation}
    \min T_f
\label{sobj}
\end{equation}

s.t. (\ref{god}), (\ref{eod}), (\ref{nlz}), (\ref{nlpi}), (\ref{tmy}), (\ref{fdy}), (\ref{1b1y}), (\ref{avltyz})

\section{Computational analysis}

\section{Conclusion and future extensions}

%% The Appendices part is started with the command \appendix;
%% appendix sections are then done as normal sections
%% \appendix

%% \section{}
%% \label{}

%% If you have bibdatabase file and want bibtex to generate the
%% bibitems, please use
%%
%%  \bibliographystyle{elsarticle-harv} 
%%  \bibliography{<your bibdatabase>}

%% else use the following coding to input the bibitems directly in the
%% TeX file.

\begin{thebibliography}{00}

%% \bibitem[Author(year)]{label}
%% Text of bibliographic item

\bibitem[ ()]{}

\end{thebibliography}
\end{document}

\endinput
%%
%% End of file `elsarticle-template-harv.tex'.
